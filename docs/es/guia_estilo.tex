\documentclass[a4paper, 10pt]{report}
\usepackage[spanish]{babel}
\usepackage{listings}
\lstset{language=Python}

\begin{document}
\title{e-cidadan\'{\i}a\\ Gu\'{\i}a de estilo para desarrolladores}
\author{Oscar Carballal Prego \texttt{<info@oscarcp.com>}}

\maketitle
\newpage 

\tableofcontents
\newpage

\chapter{Python}

\section{Imports}
	\begin{enumerate}
		\item Todos los imports deben estar situados en la cabecera del fichero, por debajo de la cabecera de comentarios.
		\item Los imports de m\'{o}dulos de sistema o de python deben preceder a los dem\'{a}s, y los de las librer\'{\i}as externas deben preceder a los de la aplicaci\'{o}n.

\begin{lstlisting}
import os
import sys

from extlib import function

from myapp.module import function
\end{lstlisting}
	
	\end{enumerate}

\section{Columnas}
El c\'{o}digo debe de ser de 80 columnas de ancho como m\'{a}ximo salvo en los casos de las plantillas de estilo.

Si una l\'{\i}nea de c\'{o}digo no cabe en 80 columnas, intenta reducirla declarando variables previamente. Si a\'{u}n as\'{\i} no se puede y la l\'{\i}nea tiene par\'{e}ntensis, debe dividirse por ese lugar. Ejemplo:

\begin{lstlisting}
website = models.URLField(_('Website'), verify_exists=True,
			 max_length=200, null=True, blank=True,
			 help_text=_('The URL will be checked'))
\end{lstlisting}

\section{Indentaci\'{o}n}
La indentaci\'{o}n debe ser de 4 {\bf espacios} por nivel. No se pueden utilizar tabulaciones para marcar los niveles de indentaci\'{o}n.

\chapter{HTML}

\section{Columnas}
El c\'{o}digo HTML no tiene l\'{\i}mite de columnas, pero debe estar indentado de forma que se pueda localizar r\'{a}pidamente cualquier elemento del documento. La disposici\'{o}n indentada en el desarrollo prevalece sobre el resultado renderizado de la aplicaci\'{o}n.

\end{document}